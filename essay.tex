% Please do not change the document class
\documentclass{scrartcl}

% Please do not change these packages
\usepackage[hidelinks]{hyperref}
\usepackage[none]{hyphenat}
\usepackage{setspace}
\doublespace

% You may add additional packages here
\usepackage{amsmath}

% Please include a clear, concise, and descriptive title
\title{Video Games in Education}

% Please do not change the subtitle
\subtitle{COMP130 - Research Journal}

% Please put your student number in the author field
\author{1605240}

\begin{document}

\maketitle


\section{Introduction}
Children are using technology from a very young age \cite{chuang2007effect} . This is why video games are becoming more popular and with that games in education have become more popular too. They are used because it makes for a more engaging way of learning which the students can relate to more. It helps them to learn if they are enjoying what they are doing and it has been shown that playing games can lead to higher test results \cite{marques2012video} and better reaction time \cite{janarthanan2012serious}, these are just some of the benefits of games as well as a general one of making kids smarter \cite{janarthanan2012serious}. I'm interested in seeing studies which prove that games aren't bad and that talk about the benefits of using games as when I was growing up I was always told that video games are bad and they make your eyes square. 


\section{Effect of Computer-Based Video Games on Children: An Experimental Study \cite{chuang2007effect}}
This paper did an experiment to see whether or not "computer-based video games facilitate children cognitive learning." Research from previous experiments has show that games can help children to learn things easier especially in science, maths and literacy. I personally believe learning is more fun when used in the form of a game as I tend to remember more and it's a more enjoyable experience when it doesn't feel like learning. The results of the experiment showed that the people who played the game knew more about the topic than the other group following a computer-assisted instruction program. They even said these games could function as powerful tools in assisting children's cognitive process. I'm not surprised by the results as children like to learn when it doesn't feel like they're being made to learn it. To make the experiment fairer they suggest they could have used a larger sample size which would have needed more participants but were limited by the computers they had and parents permission. 


\section{Video games as a medium for software education \cite{marques2012video}}
This paper talks about how there is a declining interest in the fields of engineering and computer science, however video games seem to become more popular by the day. Not surprised as there are so many jobs to do these days. They decided to try and make a game that would use visual programming to help interest and teach university students in a fun way. They gave the game to students on 3 course (game design, applied computing and electrical engineering) and found out that they rated it highly for learn-ability and memorability. Before they had all the students do a test to test their knowledge and compare after they had played the game. The results showed that electrical engineers knowledge went up by 11 percent however the applied computing students did not. This is because they already had basic programming skills. The results interest me as the engineers students mark went up a significant amount in such a short span of time and it shows that games can be used to make learning fun and helps you to remember information. Games have the ability to connect with more people as they are enjoyable.



\section{Serious Video Games: Games for Education and Health \cite{janarthanan2012serious} }
Serious games are normally used for "better health" and a better "educational environment" and they don't always mean the game will be produced properly.The writer talks about how games play a big part in learning to all demographics these days, I think this is because you can make games for everyone at different levels and still manage to make it interesting for everyone. They did experiment where two groups of surgeons did laparoscopic surgery and found that the ones who played games finished 27 percent faster and made 37 less mistakes. This shows how effective playing video games can be for your reactions, hand eye coordination and on the spot decisions which can also help to make you smarter. They also did other experiments which proved the children who played games had better results. The paper says more serious games should be made for educational use which I agree with fully because they can help you in what you're learning but make it seem as you're just having fun.

\section{PlayLOGO 3D: A 3D Interactive Video Game for Early Programming Education: Let LOGO Be a Game \cite{paliokas2011playlogo} }
This paper talks about the game PlayLOGO 3D. It's a children's game designed to teach children the beginning of programming. It is designed for the ages of 6-13 which shows that games can have uses to more than 1 age range of child. Holzinger et al said "especially small children do not make a distinction between play and learning, play and work, fantasy and reality", as long as they are having fun they don't notice. They talk about other educational games that have been created using LOGO (the programming language) used to help children learn, one of which we have used called Lego Mindstorm. Mindstorm was great as it challenged you to think about how to build the device then you actually had to program it in a simple environment which is great for learning the basics but you don't even realize you're working because it was enjoyable. They had to make sure PlayLOGO was designed in the right way so it was insightful but fun, this shows they spent a lot of time trying to design it so that it was an effective tool.

\section{Using games for teaching crisis communication in higher education and training \cite{ki2016using} }
Games aren't just used for fun, this paper talks about how the military use simulation games for training and mentions that some of these games have "spin offs" that are played by the public, I myself play some of these games are they can be very close to real life and provide a thrill. They then talk about how games can be used for "crisis communication" which gives people the knowledge of what to do in real live situations. This could help the person if they ever came into a situation since they'd know how to deal with it. The write mentions that as we get better technology there is more potential for learning games which is such a valid statement as technology like VR can be close to simulating real life and the video game industry is expanding. Even though it's a crisis game they had to make it engaging, they tried to do this by adding 2 groups "against" each other. This meant that when you play you'd want to do better than the other group so it leads to better communication. One problem that was brought up was some students couldn't keep track as so much was going on, they are looking at building a level with less going on so that the level is easier for them, this show they are trying to make it friendly for everyone to learn on.

\section{Game-based enabled e-learning model for e-Safety education \cite{underhay2016game} }
This paper is about e-Safety and how games can help keep you safe online. It talks about how students maybe unaware of internet safety and how to stay safe online. This is important as if you're using the internet you should know what information you can post where to save you from fraud. They made a game where the player plays as a system admin from the university and has to keep the network protected as well as three other games. This makes it seem more real as it is based at their university but at the same time it teaches them how to stay safe online and not get stressed when if the real thing happened. It mentions that 75 percent of students surveyed prefer to use a laptop or some sort of computer for their studies, so it is good if they all know how internet safety works and how they can protect themselves from giving away personal details.
\section{Conclusion}
To conclude I think that these papers all give very good and valid reasons as to why video games are used in education as well as many other fields such as health. I believe more educational institutes should be making use of the technology we have access to these days and create games to help students learn. It is a very modern method that not everyone would agree with but the studies show it helps to engage students. I certainly wish when I was at school we had more educational games to teach us as I had very vivid memories of the few we had. The main reasons video games should be used more often is they can be very fun and relevant and they bring a host of benefits with them too.

\bibliographystyle{IEEEtran}
\bibliography{references}

\end{document}
